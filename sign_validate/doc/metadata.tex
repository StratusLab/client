%%
%% An example for producing a StratusLab deliverable, milestone, or report.
%% 

\documentclass{../stylesheet/stratuslab}

%%
%% Define metadata for the document
%%

% Provide fixed date and year for final version.  Use the date format below.
\date{\dmyformat\today}
%\date{31 January 2010}
\newcommand{\copyrightyear}{\yearonly\today}
%\newcommand{\copyrightyear}{2010}

% Define document type: "Deliverable", "Milestone", or "Report"
\doctype{Report}

% Define the document ID.  Ex. "D3.2" for second deliverable of WP3.
\docid{D0.0}

% Define the document version.  Ex. "1.2".
\docversion{1.2}

% Define the title.  Capitalize each non-trivial word.
\title{StratusLab Report}

% Define the abstract.  It should be a good summary of the reports material and conclusions.
\abstract{%
This document defines StratusLab Core Metadata and how to sign and validate it using Java XML Digital Signature API
}

\begin{document}

\maketitle

% CONTRIBUTORS LIST
% \contributor{Last Name, First name}{Organization}{contributed to ... (use references, i.e., \ref{}

\begin{contributors}
  \contributor{Mohammed AIRAJ}{CNRS-LAL}{All}
\end{contributors}

%
% REVISION HISTORY
%
\begin{history}
  \historyitem{0.1}{13 May 2010}{Initial version for comment.}
\end{history}

%
% FRONT MATTER (remove lists of figures/tables if there are none)
%
\tableofcontents
\listoftables

\chapter{Introduction}
\label{sec:Introduction}

First part of this document defines StratusLab Core Metadata (SCM). Elements of SCM were then mapped to Dublin Core Metadata (DCM) elements, and used with RDF/XML. \\
Second part show how to sign and validate SCM files using Java XML Digital Signature, with both grid certificates and RSA/DSA key pair.   

\section{StratusLab Core Metadata}
\label{sec:DocumentMetadata}
\begin{table}
\caption{StratusLab Core Metadata}
\label{tab:DocumentMetadata}
\begin{center}
\begin{tabular}{l>{\raggedright}p{0.65\linewidth}}
\hline
\textbf{Metadata} &  \textbf{Comment} \T\B\tabularnewline
\hline
\T cretaed & image time creation, should respect this format : yyyy-mm-dd  hour:min:sec \tnl

type & image type, could be base or grid \tnl

version & image version  \tnl

arch &  Image Operating System Architecture\tnl

user & User who create this images \tnl
os  & Operating Sytem \tnl
osversion & Operating System version\tnl
compression & image compression format\tnl
filename & name given to the image \tnl
cheksum &  md5 and sha1 cheksum of the image\tnl
comments & comments on how to deploy the image \tnl

\hline
\end{tabular}
\end{center}
\end{table}

\newpage 

\begin{table}
\caption{Dublin Core Metadata,\\ source : http://dublincore.org/documents/dces/}
\label{tab:DocumentMetadata}
\begin{center}
\begin{tabular}{l>{\raggedright}p{0.65\linewidth}}
\hline
\textbf{Metadata} &  \textbf{Comment} \T\B\tabularnewline
\hline
\T contributor & An entity responsible for making contributions to the resource. \tnl

coverage & The spatial or temporal topic of the resource, the spatial applicability of the resource, or the jurisdiction under which the resource is relevant. \tnl

creator & An entity primarily responsible for making the resource. \tnl

date &  A point or period of time associated with an event in the lifecycle of the resource.\tnl

description & An account of the resource. \tnl
format  & The file format, physical medium, or dimensions of the resource.\tnl
identifier & An unambiguous reference to the resource within a given context.\tnl
langage & A language of the resource.\tnl
publisher & An entity responsible for making the resource available. \tnl
relation &  A related resource.\tnl
rights & Information about rights held in and over the resource. \tnl
 source &  A related resource from which the described resource is derived. \tnl
subject  &  The topic of the resource.\tnl
Title   &  A name given to the resource.\tnl
 Type   &  The nature or genre of the resource.\tnl
\hline
\end{tabular}
\end{center}
\end{table}

\newpage


\begin{table}
\caption{Mapping StratusLab Core Metadata to Dublin Core Metadata}
\label{tab:DocumentMetadata}
\begin{center}
\begin{tabular}{l>{\raggedright}p{0.65\linewidth}}
\hline
\textbf{StratusLab Core Metadata} &  \textbf{Dublin Core Metadata} \T\B\tabularnewline
\hline
\T cretaed & date \tnl

type & type \tnl

version & -  \tnl

arch &  -\tnl

user & creator \tnl
os  & description\tnl
osversion & -\tnl
compression & format\tnl
filename & title \tnl
cheksum &  could be used with identifier\tnl
comments & description \tnl

\hline
\end{tabular}
\end{center}
\end{table}


os could be mapped to  DC description element, and defined with  arch, version, format and title in one one element. Operating system for example. 

\newpage

{\bf Example :} 
\begin{verbatim}
<rdf:RDF xmlns:rdf="http://www.w3.org/1999/02/22-rdf-syntax-ns#"
 xmlns:dcterms="http://purl.org/dc/terms/">

 <rdf:Description rdf:about="http://appliances.stratuslab.org/images">
  <dcterms:subject rdf:resource="http://appliances.stratuslab.org/images/base/centos-5.5-i386-base/1.0"/>
  <dc:title>centos base image</dc:title>
  <dc:type>base</dc:type>
  <dc:date>2010-07-31</dc:date>
  <dc:creator>Charles Loomis</dc:creator>
  <dc:publisher/contributor>StratusLab</dc:publisher>
  <OperatingSystem>
     <Info>Specifies the operating system</Info>
     <Arch>i386</Arch>
     <dc:description>CentOS</dc:description>
     <dc:version>5.5</dc:vesrion>
     <dc:format>gz</dc:format>
     <dc:title>centos-5.5-i386-base-1.0.img.gz</dc:title>
  </OperatingSystem>
  <dc:identifier type="md5">e3a7d06ad2a79b2f57d9550e7436bfa4</dc:identifier>
  <dc:identifier type="SHA1">7ac5b9ac1456b89201ef2adda376e181ece060df</dc:identifier>
  <dc:description>
   Uses standard StratusLab contextualization.
   Image only has 'root' account configured.
   Only logins via ssh keys are allowed.
  </dc:description>
 </rdf:Description>
</rdf:RDF>
\end{verbatim}

\section{Signing and Validating StratusLab Metadata files}
For signing and validating metadata files we are using XML Digital Signature API. \\
Metadata files could be signed using p12 Grid Certificates or DSA/RSA private keys. \\
Verification and Validation automatically detects signature algorithm and type of private key used for signing metadata files, verify metadata file and print DN of user who signed the metadata file ( In case of grid certificates). \\

Codes : \\
GenXmlSign.java for signing metadata files \\

Usage: java GenXmlSign [medatadalfile] [outputfile] [P12 GridCertificate] [P12 GridCertificate Passwd] \\

ValidateSign.java to verify and validate metadata file. \\

Usage : java ValidateSign signedmetadatafile \\

centos.xml : metadatfile for testing


\end{document}
